\typeout{ ====================================================================}
\typeout{ this is file letter.tex, created at 13-Nov-2014               }
\typeout{ maintained by Gustavo Rabello dos Anjos                             }
\typeout{ e-mail: gustavo.rabello@gmail.com                                   }
\typeout{ ====================================================================}

\documentclass[12pt,a4paper]{article}
\usepackage[utf8]{inputenc}
\usepackage[english]{babel}
\usepackage[top=3cm,bottom=3cm,left=4cm,right=4cm]{geometry}
\usepackage{graphicx}    % pacote para inclusao de figuras,
\usepackage{subfig}
\pagestyle{plain}


\begin{document}

\begin{tabular}{ccc}
 \begin{minipage}[c]{6cm}
  \begin{flushleft}
   \includegraphics[scale=0.055]{figs/uerj-bw.png}\\
   \includegraphics[scale=1.0]{figs/gesar.png}
  \end{flushleft}
 \end{minipage}
&
&
	\begin{minipage}{6.5cm}
		\begin{flushleft}
			\small
			\noindent GESAR/UERJ \\
			Rua Fonseca Teles, 121 \\ 
			S\~ao Crist\'ov\~ao \\ 
			Rio de Janeiro -- Brazil\\
			tel.: +55 21 2332-4733\\
			{\tt gustavo.anjos@uerj.br}\\
			{\tt http://www.gesar.uerj.br}\\
			\vspace*{2mm}
			19th. November 2014
		\end{flushleft}
	\end{minipage}
\end{tabular}

\vspace*{1.5cm}
\noindent \textbf{To}: University of Minnesota, Graduate School.\\
\noindent \textbf{Subject}: Reference letter.\\

\noindent Dear Sir or Madam:\\

I am writing this letter for Mr. Eduardo Vitral Freigedo Rodrigues
regarding his work under my orientation at the GESAR lab. I confirm
that Eduardo has been working in the GESAR lab as a master student for
about 5 months period and he plans to finish his master dissertation in
July, 2015. He was granted with a master student scholarship which will
completely cover his studies at the State University of Rio de
Janeiro.

I met Mr. Eduardo Vitral while he was an undergraduate student at the
Metallurgy and Materials Engineering Department in the Federal University
of Rio de Janeiro -- UERJ/COPPE, in 2012. He was an outstanding student,
showing excellent intellectual skills, great interest in research, and
ranking first in his class.

Currently, Mr. Eduardo Vitral is following specific courses to improve
his knowledge in material science and numerical methods. He has
successfully accomplished a challenging course of Finite Element Method,
where he developed, in a very short period of time, a two-dimensional
stokes equation simulator with pressure and velocity coupling using the
knowledge acquired during the finite element course. He aims to follow
courses of Continuum Mechanics, Theory of Elasticity and Stochastic
Methods during the first semester of 2015. He is currently following
three courses namely Pattern Formations, Advanced Calculus and Numerical
Methods for Engineering (ending late 2014). I believe it can ensure
the basis of a solid education for Eduardo, preparing him for a
successful Ph.D. program. 

While in the GESAR lab, he has taken the opportunity to get well rounded
on different topics beyond his own work, besides his relatively short
period in the lab. In particular, I very much enjoy his social
participation in the GESAR lab, discussing new ideas with colleagues,
supporting lab workspace improvements and helping others with different
tasks. He speaks English and French very well, he is articulate, and he
is a diligent and motivated person. He has shown good writing skills
while preparing presentations, reports and manuscripts, and he also
aims to writing his thesis in English, although not mandatory at the
University of Rio de Janeiro.

In summary, I wish to give my very, very strong recommendation to Mr.
Eduardo Vitral for his future career in research. I have absolutely no
doubt that he will excel and that we will hear much more from him in the
future.

Yours Faithfully,

\vspace{2cm}

\begin{center}
Prof. Gustavo Anjos
\end{center}

\end{document}


\typeout{ ****************** End of file letter.tex ****************** }

